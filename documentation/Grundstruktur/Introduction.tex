\chapter{Introduction}

The rise of the Internet of Things (IoT) has brought forth a paradigm shift in the way industries operate, allowing for unprecedented levels of automation, connectivity, and data-driven decision-making. In particular, the implementation of IoT in smart factories has revolutionized manufacturing processes, enabling enhanced efficiency, predictive maintenance, and real-time monitoring. As these smart factory environments become increasingly interconnected, the need for a robust and secure IoT framework is crucial to ensure the integrity, availability, and confidentiality of data and operations.

This paper explores the utilization of AWS IoT Core, an advanced IoT service provided by Amazon Web Services, as a framework for implementing IoT solutions in a smart factory context. AWS IoT Core offers a comprehensive suite of features and functionalities designed to address the specific challenges and requirements of industrial IoT deployments. By leveraging the capabilities of AWS IoT Core, organizations can harness the power of the cloud and IoT technologies to transform their factories into intelligent, interconnected ecosystems.

The paper begins with an overview of the assignment, highlighting the objectives and scope of the paper. Subsequently, it delves into various use cases that exemplify the application of AWS IoT Core in different aspects of a smart factory. These use cases, including factory gate control, lifting platforms, hydraulic presses, and factory lighting systems, illustrate the breadth of possibilities that AWS IoT Core offers in optimizing operations and improving productivity.

Following the exploration of use cases, the paper examines the core components and functionalities of AWS IoT Core. Specifically, it explores the capabilities of AWS IoT Core in device management, communication protocols, integration with other AWS services, and the robust security measures it provides. Understanding these features is essential for effectively leveraging AWS IoT Core in building secure, scalable, and reliable IoT solutions for smart factories.

The paper also delves into the implementation aspects of AWS IoT Core, providing insights into the practical steps involved in deploying and configuring the framework within a smart factory environment. Additionally, it introduces the concept of an IoT simulator, which facilitates testing, validation, and simulation of IoT deployments, enabling organizations to refine their solutions before actual implementation.

In conclusion, this paper serves as a comprehensive exploration of the utilization of AWS IoT Core as an IoT framework for smart factories. By leveraging the capabilities of AWS IoT Core, organizations can unlock the full potential of IoT technologies and cloud computing, enabling them to optimize operations, improve efficiency, and drive innovation in the ever-evolving landscape of smart manufacturing.

\section{Assignment}
Setzen Sie den zugewiesenen Use Case um.
Finden Sie dafür ein geeignetes IoT Framework und begründen Sie dessen Eignung für den jeweiligen Case. In unserem Fall \ac{aws}.\\
\\
Inhalt der Seminararbeit (max. 10 Seiten):
\begin{itemize}
    \item Beschreiben Sie das IoT-Framework, Ihr Lösungskonzept und die gewählte technische Umsetzung
    \item Gehen Sie dabei auf Herausforderungen aus technischer und konzeptioneller Sicht sowie auf Security ein
\end{itemize}

\begin{tabbing}
	Abgabe: \hspace{3em} \= 31.05.2023 \hspace{.3em} \= via Moodle (Seminararbeit + Präsentation)\\
	Präsentation: \> 02.06.2023 \> Demonstration des Use Cases erwünscht!
\end{tabbing}

%Zitat \autocite[S. 20]{diekmann2010}


