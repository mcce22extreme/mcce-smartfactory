\chapter{Used Framework}

\section{AWS IoT Core}
We chose the IoT Framework AWS IoT Core to implement the task. 
AWS IoT Core is a cloud-based platform provided as part of the \ac{aws} cloud.
This IoT framework enables secure communication and data exchange between connected devices, cloud applications and other endpoints.
The platform acts as a central communication hub for the distribution, management and monitoring of all connected IoT devices and applications.

\subsection{Device Management}

IoT devices can be added to AWS IoT Core using a device configuration. 
Each registered IoT device is assigned a unique identifier, which is then used to authenticate and authorize the device to interact with AWS IoT Core and in further succession with other applications or device.

By defining device shadows, physical devices can be virtually represented within AWS IoT Core. 
With the help of TwinMaker, another IoT service from AWS, so-called digital twins of a physical \ac{iot} device can be created ({\cite{ref01}}).
In contrast to a digital shadow, a digital twin allows a even closer connection between the digital and physical worlds where data flows in both directions.
This virtual representations of a physical device allows applications to interact with the devices and retrieve its current status and configurations even if the devices are not currently connected to the Internet. 

In addition to the above configuration options, AWS IoT Core also provides device management tools such as device monitoring, device authentication and authorization, and device analytics to monitor and manage IoT devices throughout their lifecycle ({\cite{ref02}}).

\subsection{Communication}

AWS IoT Core supports the following 3 communication protocols:

\begin{itemize}
    \item \ac{mqtt}
    \item \ac{https}
    \item WebSockets
\end{itemize}
Communication via \ac{mqtt} uses a subscriber messaging model that is commonly used in \ac{iot} applications. 
Messages are exchanged via a centrally operated message broker, whereby applications can only subscribe to those message types, so-called topics, in which they are interested.

AWS IoT Core also supports \ac{https} communication through a RESTful API that allows \ac{iot} devices and cloud applications to securely send and receive messages over \ac{http}.

Last but not least, WebSockets can also be selected for communication. WebSockets offers a bidirectional communication protocol that enables real-time communication between \ac{iot} devices and cloud applications. 
It creates a persistent, low-latency connection between the two endpoints that allows data to be exchanged in near real-time ({\cite{ref02}}).

\subsection{Integration}

By using common communication protocols such as \ac{mqtt}, \ac{https} or WebSockets, AWS IoT Core can already be easily integrated into existing \ac{iot} infrastructures. 
In addition to these protocols, AWS IoT Core provides its own \ac{sdk}s for popular programming languages such as C, Java, Python, and Node.js that help make it easier to connect and communicate with IoT devices. 
These \ac{sdk}s implement the AWS IoT Device Gateway protocol, which is a secure and scalable method of transferring data to \ac{iot} devices over \ac{mqtt}, \ac{https} or WebSockets.
In addition to integration and interaction with other \ac{iot} devices, a variety of other AWS services such as Lambda functions, DynamoDB, \ac{sns}, \ac{sqs}, \ac{s3}, etc. can be controlled by defining so-called message rules (\cite{RefAwsIotCoreFeatures}).

\subsection{Security}

AWS IoT Core provides several security features and capabilities to ensure the secure and reliable operation of Internet of Things (IoT) applications.
The key security aspects of AWS IoT Core are:

\begin{itemize}
	\item \textbf{Authentication and Authorization:} AWS IoT Core uses the AWS Identity and Access Management (IAM) service to manage user identities and access control. IAM allows you to define granular access policies and roles to control who can perform actions on your IoT devices and data.
	
	\item \textbf{Device Authentication:} Devices connecting to AWS IoT Core are authenticated using X.509 certificates or Amazon Cognito user pools. This ensures that only authorized and trusted devices can communicate with the IoT Core service.
	
	\item \textbf{Device Authorization:} AWS IoT Core supports fine-grained authorization policies that determine what actions a device can perform on specific resources. You can define rules based on device attributes, such as device ID or type, and specify the allowed actions (e.g., publish, subscribe, receive messages) for each device.
	
	\item \textbf{Message Encryption:} AWS IoT Core ensures end-to-end encryption for message communication. When devices publish messages, they can use Transport Layer Security (TLS) to encrypt the data in transit. The messages can also be encrypted at rest using the AWS Key Management Service (KMS).
	
	\item \textbf{Secure Communication:} AWS IoT Core supports MQTT, MQTT over WebSocket, and HTTP protocols for secure communication between devices and the cloud. MQTT messages can be sent over TLS, providing a secure and efficient communication channel.
	
	\item \textbf{Device Shadows:} Device Shadows provide a virtual representation of your IoT devices in the cloud, allowing applications to interact with the devices even when they are offline. Device Shadows use TLS for secure communication, and access to shadows can be controlled through IAM policies.
	
	\item \textbf{Fleet Provisioning:} AWS IoT Core offers fleet provisioning capabilities, which allow for secure and scalable onboarding of large numbers of devices. You can use device certificates, IoT Device Management, or custom provisioning templates to securely register devices with AWS IoT Core.
	
	\item \textbf{Rule-Based Access Control:} AWS IoT Core enables you to define rules to perform actions on incoming messages, such as filtering, transforming, or routing them to other AWS services. You can control access to these rules using IAM policies to ensure that only authorized entities can modify or execute them.
	
	\item \textbf{Device Management:} AWS IoT Core integrates with AWS IoT Device Management, which provides features like device provisioning, over-the-air updates, and remote device management. These features help maintain the security and health of your IoT devices throughout their lifecycle.
	
	\item \textbf{Logging and Monitoring:} AWS IoT Core integrates with Amazon CloudWatch, allowing you to collect and analyze logs and metrics related to your IoT devices and applications. You can set up alarms and notifications based on predefined thresholds to detect and respond to security events or anomalies.
\end{itemize}

It's important to note that these are some key security aspects provided by AWS IoT Core, but security is a shared responsibility. 
AWS provides a secure foundation, but it's crucial to implement security best practices, such as securing device endpoints, regularly updating firmware, and implementing secure coding practices ensuring the overall security of your IoT solution (\cite{RefAwsIotCoreSecurity}, \cite{RefAwsIotCoreSecurityBestPractices}).